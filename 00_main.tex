\documentclass[12pt]{article}
\usepackage[utf8]{inputenc}
%\usepackage[english]{babel}

\usepackage{varwidth}
\usepackage{lipsum}
\usepackage{xurl}
\usepackage{graphicx}
\usepackage{nameref}
\usepackage{acronym}
\usepackage{mathptmx}
\usepackage[margin=2.5cm, tmargin=3cm, headheight=1cm]{geometry}


%\usepackage{csquotes}
\usepackage[style=apa, natbib=true]{biblatex}
\addbibresource{literature.bib}

%\usepackage[hidelinks]{hyperref} % without colored boxes around links
\usepackage{hyperref} % with colored boxes around links -- useful for proofreading but less recommended for submission
\usepackage{float}

\setlength{\parskip}{0.5\baselineskip}%
\setlength{\parindent}{0pt}%


\widowpenalty=10000
\clubpenalty=10000
\sloppy

\renewcommand{\baselinestretch}{1.5}

\title{\huge\textbf{A taxonomy of centralization dynamics in the OSI layers} }
\author{\Large
Jonas Acum, Student ID, \href{mailto:author1@uni-muenster.de}{author1@uni-muenster.de} \\
\Large Tom Skirde, 527372, \href{mailto:tom.skirde@uni-muenster.de}{tom.skirde@uni-muenster.de} }

\date{}

\begin{document}

\renewenvironment{abstract}
{
	\clearpage
	\pagenumbering{gobble}
	\thispagestyle{empty}
	\small
	\begin{center}
		\bfseries\normalsize \abstractname\vspace{-1em}\vspace{-1em}
	\end{center}
	\rule{\textwidth}{1pt}
	\vspace{-1.5em}
	\list{}{%
		\setlength{\leftmargin}{8mm}
		\setlength{\rightmargin}{\leftmargin}%
	}%
	\item\relax}
{\endlist \vspace{-1em}	\rule{\textwidth}{1pt}} 

\raggedbottom

\begin{figure*}
\begin{center}
\includegraphics[width=0.50\linewidth, trim=0cm -0.5cm -3cm 0cm, clip]{Graphics/Icon_Uni_Muenster.pdf}
\includegraphics[width=0.30\linewidth, trim=0cm 3cm 0cm 0cm, clip]{Graphics/Icon_ERCIS.pdf}
\end{center}
\end{figure*}
\vspace{2cm}
\begin{center}
    \Large \textbf{Essay} \\[0.5cm]
    Security of Distributed Systems \\[0.5cm]
    Summer Semester 2025
\end{center}
\vspace{1cm}
{\let\newpage\relax\maketitle}
\vspace{1cm}
\begin{center}
    \large Lecturer: Dr. Johannes Sedlmeir
\end{center}
\vspace{0.2cm}
\begin{center}
    \large Submission date: \today
\end{center}

\thispagestyle{empty}

\clearpage~
\thispagestyle{empty}

\clearpage%
\begingroup%
\thispagestyle{empty}%
\null\vfill%
\centering%
\begin{varwidth}{\textwidth}
    {\raggedright\large\itshape%
    Any sufficiently advanced technology is indistinguishable from magic.\par\bigskip%
    }%
    {
    \raggedleft\large\scshape{Arthur C. Clarke}\par%
    }
\end{varwidth}%
\vfill\vfill%
\endgroup%
\clearpage
\thispagestyle{empty}
~
\vspace{4cm}
\begin{abstract}
    A short summary of the essay (max. 200 words). \lipsum[1] \\[0.5cm]
    \textbf{\emph{Keywords:}} Keyword 1, ...
    
\end{abstract}
\vspace{2cm}
\begin{figure}[H]
\includegraphics[width=2cm]{Assets/cc-by.png}
\end{figure}
\vspace{-0.75cm}
This work is licensed under the license \href{https://creativecommons.org/licenses/by/4.0/legalcode}{Attribution 4.0 International}.
\thispagestyle{empty}
\clearpage
\renewcommand{\contentsname}{Table of Contents}
\tableofcontents
\thispagestyle{empty}
\clearpage
\section*{Optional: Acronmys}
\begin{acronym}[GDPR] % <-- put longest acronym here for good indentation
\acro{ZKP}{zero-knowledge proof}
\end{acronym}
Note: With \verb+\acrodefplural+ you can define custom plural. Refer to acronyms with \verb+\ac{}+ and \verb+\acp{}+. Sometimes you may need the capitalized versions: \verb+\Ac{}+. \verb+\Acp{}+ is sometimes buggy - if you need this and would like to include an index, consider using the ``glossaries'' package (I can help you out there).
\section*{Optional: List of Figures and List of Tables}
\listoffigures
\listoftables
\thispagestyle{empty}
\setcounter{page}{0}
\clearpage

\pagenumbering{arabic}
\section{Introduction}
\label{sec:Introduction}

To create a common basis for discussion, it is important to first define the terms \textit{digital identity} and \textit{subject}. A digital identity is often defined as a digital reference to a person~\citep{domingo2020SSI}. It is thus something that a subject has and uses in response to requests for digital identification, authentication, or proofs of authorisation. A digital identity consists of attributes that can be revoked, deleted, transferred, or exchanged, such as citizenship, institutional affiliations, and proofs of ownership~\citep{lyons2019blockchain,preukschat2021why}. Identity attributes are typically connected to a subject via a unique identifier within a system or domain, such as an index in a database or a social security number~\citep{bosworth2005entities}. 

Some new paragraph ...\lipsum[2]

Another paragraph .... \lipsum[3]
\clearpage
\section{Background}
\label{sec:Background}

As highlighted in the \nameref{sec:Introduction} Section / Section~\ref{sec:Introduction}, ...

\subsection{The OSI reference model}

For a template, visit \url{https://www.overleaf.com/read/xwhrcxycrsmh} or see the zip file.

If you write the paper with MS~Word, consider using reference management software that provides a Word plugin, e.g., Citavi. For \LaTeX, one can use Biblatex directly or use tools like Zotero.

\Acp{ZKP} are important. A \ac{ZKP} is defined via ... We use \acp{ZKP} in ...


\subsection{The end-to-end principle}

\begin{figure}[!tb] % <-- Specify priority to include top, bottom, or here
    \centering
    \includegraphics[width=2cm, trim=0cm 2cm 0cm 0cm, clip]{Graphics/Icon_ERCIS.pdf}
    \caption{My figure caption}
    \label{fig:my_label}
\end{figure}

Figures can be cropped conveniently. One can link to them conveniently; see Figure~\ref{fig:my_figure_label}. If you make heavy use of referring to figures and sections, the ``cleveref'' package may come in handy. Vector images (PDFs are easiest to handle) are preferred. One good approach is making the Figures in Powerpoint, exporting them as PDF file (multiple pages), uploading them as Figures.pdf and then use \verb+\includegraphics[page=1, width=..., trim=..., clip]{Figures.pdf}+. This avoids the need to upload many individual figures. For instance, I uploaded an example presentation and included the second and third page in the following two figures.

\begin{figure}
    \centering
    \includegraphics[page=2, width=0.8\linewidth]{Figures/Example presentation.pdf}
    \caption{Second page of the presentation.}
    \vspace{2em}
    \includegraphics[page=3, width=0.8\linewidth]{Figures/Example presentation.pdf}
    \caption{Third page of the presentation.}
    \label{fig:my_figure_label}
\end{figure}

\subsection{Forces of centralization}
Use [H] to place the table at exactly this place. Use [h] to place it at this place if suitable.  Most of the time, [!tb] will be better, then it is placed at the top or bottom of a page nearby. Refer to table via Table~\ref{tab:my_table_label}.
\begin{table}[H]
    \centering
    \begin{tabular}{c|c}
        A & B \\\hline
        C & D
    \end{tabular}
    \caption{My table caption.}
    \label{tab:my_table_label}
\end{table}
\clearpage
\input{03_Methodology}
\clearpage
\input{04_State_of_Research}	
\clearpage
\section{The Taxonomy of Centralization}
\label{sec:The_Taxonomy_of_Centralization}




Consistency in reference formatting is key. The following guidelines are \textbf{*not*} mandatory. But they will give you a feeling that uniformity in the references style has many facets that you should consider.


%############################4.1 
\subsection{Centralization at the physical layer (L1)}
\textbf{The only things I mandate are:}  
\begin{itemize}
    \item To use an \emph{authoryear} citation style (as opposed to numeric), as it helps readers to quickly recognize papers previously encountered.
    \item To include a \texttt{DOI} or \texttt{URL} for every reference that is not a published book or there are good reasons why there is no online pointer for it
\end{itemize}
Both settings are already active in this template (provided the DOIs / URLs are specified in the references in literature.bib).




%############################4.2
\subsection{Centralization at the data link layer (L2)}


%############################4.3
\subsection{Centralization at the network layer (L3)}


%############################4.4
\subsection{Centralization at the transport layer (L4)}

%############################4.5
\subsection{Centralization at the session \& presentation layers (L5 \& L6)}

%############################4.6
\subsection{Centralization at the application layer (L7)}

%############################4.7
\subsection{Cross-layer view}
\clearpage
\input{06_Cross-Layer_Analysis}
\clearpage
\section{Discussion}
\label{sec:Discussion}

%#############################7.1
\subsection{Theoretical and practical implications}

\subsubsection{Implications for network design}

\subsubsection{Implications for policy and regulations}


%#############################7.2
\subsection{Limitations}
The OSI model is a conceptual framework and does not perfectly map to real-world implementations, which may lead to oversimplifications in the analysis of centralization dynamics.

%#############################7.3
\subsection{Future research directions}
\clearpage
\input{08_Conclusion}


\clearpage
\printbibliography

\clearpage
\thispagestyle{empty}
\section*{Declaration of authorship}
I/We hereby declare that, to the best of my knowledge and belief, this essay titled ``XXX'' is my/our own work. I/We confirm that each significant contribution to and quotation in this thesis that originates from the work or works of
others is indicated by the proper use of citations and references.

Münster, \today \\[2cm]
Firstname Lastname Author 1 \qquad\qquad\qquad
(Firstname Lastname Author 2)

\clearpage
\thispagestyle{empty}
\section*{CRediT author statement (only for more than one author)}

You can use this \href{https://www.elsevier.com/researcher/author/policies-and-guidelines/credit-author-statement}{Link to Elsevier's Contributor Roles Taxonomy} as guideline. Please fill it out for every Section and for longer subsections. Of course, you can also choose other forms, such as a paragraph for each author that details the individual contributions.

\clearpage
\thispagestyle{empty}
\small
\section*{Consent form}
~\\[-3cm]
\subsection*{for the use of plagiarism detection software to check my/our essay}

\textbf{What is plagiarism?} \\
Plagiarism is defined as submitting someone else’s work or ideas as your
own without a complete indication of the source. It is hereby irrelevant whether the work of
others is copied word by word without acknowledgment of the source, text structures (e.g., line of
argumentation or outline) are borrowed or texts are translated from a foreign language.

\textbf{Use of plagiarism detection software} \\
The supervisor may use plagiarism software to check each submitted essay for plagiarism. For that purpose, the thesis would be electronically forwarded to a software service provider, where the software checks for potential matches between the submitted work and work from other sources. For future comparisons with other essays,
your essay will be permanently stored in a database. Only the School of Business and Economics
of the University of Münster is allowed to access your stored essay. The student(s) agree(s) that his or
her (their) essay may be stored and reproduced only for the purpose of plagiarism assessment. The supervisor of the essay will be advised on the outcome of the plagiarism assessment.

\textbf{Sanctions} \\
Each case of plagiarism constitutes an attempt to deceive in terms of the examination regulations and will lead to the module being graded as ``failed''. This will be communicated to the examination office, where your case will be documented. In the event of a serious case of deception, the examinee can be generally excluded from any further examination. This can lead
to the expulsion of the student. Even after graduation from university, plagiarism can result in the withdrawal of the awarded academic degree.

I confirm that I have read and understood the information in this document. I agree with the procedure described for plagiarism assessment and potential sanctioning. \\[-0cm]

Münster, \today \\[1.5cm]
Firstname Lastname Author 1\qquad\qquad\qquad (Firstname Lastname Author 2)

\end{document}
